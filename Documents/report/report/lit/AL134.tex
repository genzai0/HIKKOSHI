\documentclass[uplatex]{jsarticle} %titlepage
\usepackage[dvipdfmx]{graphicx}
\usepackage{float}

\title{アクティブラーニング13.4}
\author{C0118005 A3 秋本 遥基}
\date{\today}

\begin{document}
\maketitle

\section{Google Drive}

  保存した内容に対して、プライパシーポリシーに則った所有権、著作権が適応される(これにはGoogle者による保存内容に対する検閲の類も含まれる)。これについて様々にユーザーが保存したデータに対して危惧される意見があるが、同社の企業規模・社会的貢献から表立って著作権や所有権を利用することはないと思われる。但し、利用規約にあるとおりに悪意のあるサービスの利用についてはこの限りではない。また、一般に同社のポリシーから情報収集は同社のサービス提供のために使われるものと示しているため、例えば絵描きの著作権を犯してGoogleDriveに保存された絵画の無断使用などはない考えられる。(参照:\cite{source}\cite{pli})

\section{Cloud Garage}

  保存した内容に対して、権利一般は所有者に帰属する。追加でサポートを受けられるが、それ以外での検閲のようなものはない(参照:\cite{cloud})。クラウドサービスがオンラインストレージであるか疑問にも思ったが、情報の保存という点で違いはないと考えられる(参照:\cite{tigai})。

\begin{thebibliography}{9}
  \bibitem{source} \verb+https://wired.jp/2012/04/27/your-google-drive-files-now-in-googles-promo-materials-ars/,『Google Drive』のファイルは誰のもの?,+参照(\today)
  \bibitem{pli} \verb+https://policies.google.com/privacy,ポリシーと規約,+参照(\today)
  \bibitem{cloud} \verb+https://cloudgarage.jp/,CloudGarage,+参照(\today)
  \bibitem{tigai} \verb+https://boxil.jp/mag/a707/https://boxil.jp/mag/a707/,【ファイルサーバ×クラウドストレージ】曖昧な2つの違いを調査して明確に定義しました,+参照(\today)
\end{thebibliography}


\end{document}
