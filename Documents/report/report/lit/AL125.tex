\documentclass[uplatex]{jsarticle} %titlepage
\usepackage[dvipdfmx]{graphicx}
\usepackage{float}

\title{アクティブラーニング12.5}
\author{C0118005 A3 秋本 遥基}
\date{\today}

\begin{document}
\maketitle

\section{検知が難しいとき}

  カメラで認識した画像にもよるが、「渋滞などで車線を示すもの(白線)の明瞭な判定ができないとき」と
  「雨などによって認識した画像全体にノイズが入ってしまい、認識が難しくなる」、「検出したい白線以外の線と認識できてしまうものが存在している」「逆光やトンネル内など光が正常に認識できないとき」が挙げられる。

\begin{thebibliography}{9}
  \bibitem{susan} \verb+https://www.denso-ten.com/jp/gihou/jp_pdf/24/24-4J.pdf,画像処理による走行環境認識+
\end{thebibliography}


\end{document}
