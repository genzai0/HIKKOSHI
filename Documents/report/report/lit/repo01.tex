\documentclass[uplatex]{jsarticle} %titlepage
\usepackage[dvipdfmx]{graphicx}
\usepackage{float}

\title{薄型ディスプレイの実現方法}
\author{C0118005 A3 秋本 遥基}
\date{}

\begin{document}
\maketitle


\section{液晶ディスプレイ以外の薄型ディスプレイの実現方法について}

  以下引用文は全て「キャノンサイエンスラボ」のwebページを参考にした。\cite{sour}\\

\subsection{プラズマディスプレイ}

\begin{quotation}

  液晶ディスプレイは大型化が進み、40型を超える大画面テレビが一般に普及するようになっています。80型を超えるものも実用化されるようになってきました。
液晶ディスプレイのように、薄型で大画面を実現するフラットパネルディスプレイ(FPD)には、プラズマディスプレイもあります。プラズマディスプレイは、放電現象で光る蛍光灯の原理を応用しています。
プラズマディスプレイのパネルのセルには、キセノンなどの希ガス元素が封入されています。セルに形成されている電極に電流が流れて電子が放出されると、気体の原子と衝突、原子核から電子が離れて不安定な励起状態が起こります。励起状態からもとの基底状態に戻ろうとするとき、エネルギーが光となって出てきます。
この光は紫外線なので、白色光ではありません。パネルの各セルは3つに区切られていて、RGBの3色にそれぞれ発光する蛍光体が塗られています。蛍光体が発色して、カラー表示される仕組みです。色の調整は光の強さで行い、RGB3色が均一に発光すると白になります。RGBが光らないと黒になります。

\end{quotation}

以上は参考ページ\cite{sour}の\textit{いろいろな方式のディスプレイ(プラズマディスプレイ)}の節の引用である。
蛍光灯と同様に、封入された気体の電流による反応で光を放出する。液晶ディスプレイとの違いとして、光を放つ時点で蛍光体自身に発光の強度を指定する点が挙げられる。

\subsection{有機EL}

\begin{quotation}

  ディスプレイの発光原理は、方式によってそれぞれです。液晶ディスプレイは、利用材料(液晶)とは別にバックライトの白色光源が必要です。ブラウン管は、加速させた電子を利用して光らせます。プラズマディスプレイは、放電による紫外線を使います。LEDのように、性質の異なる材料の組み合わせに電流を流してRGBの3色を発光することができれば、わずかな電気エネルギーで素子が自発光するディスプレイができるかもしれません。
自発光型ディスプレイとして期待されているのが、有機EL(Organic Light Emitting Diode:OLED)ディスプレイです。材料には「有機化合物」を使います。有機化合物とは、炭素(C)を含む化合物全般(CO、CO2などは除く)のことで、身近な有機化合物の代表にはプラスチックがあります。有機化合物以外の無機化合物を材料にして無機ELディスプレイもできますが、直流では長時間安定動作できないため、現在は、直流低電圧で駆動できる材料が見つかっている有機ELディスプレイの実用化が始まっています。

  有機ELディスプレイの構造は、下図のようになっています。両端の電極(マイナス極・プラス極)に電圧をかけると、マイナス極から出る電子は電子輸送層の分子によって、発光層に注入されます。一方プラス極の側からは、電子が抜け出した“穴”である「正孔(せいこう)」が正孔輸送層の分子によって発光層に注入されます。発光層では、電子と正孔が再結合して励起状態となり、もとの基底状態に戻るとき、光が発生します。発光する色は、材料に使われている物質が発する光の波長によって決まります。
有機ELディスプレイの材料には、さまざまな物質が試されてきました。現在、携帯電話などの小型ディスプレイや一部テレビでの実用化が始まり、平面照明という新たな照明器具として商品化されています。大画面・フレキシブルなディスプレイへの対応を視野に、材料や製造方法の探求が盛んに行われているところです。

\end{quotation}

以上は参考ページ\cite{sour}の\textit{いろいろな方式のディスプレイ(有機EL)}の節の引用である。
陰極と陽極からそれぞれ、電子と正孔が注入されその先の任意の発光層で光を放出する方法である。液晶ディスプレイとの違いとして、引用の通り、白色光源が不要で放電による紫外線のみで光を放つことができることである。同じく引用の通り、少ない電力でディスプレイを使用することができるので小型端末への実用性が見込まれている。

\section{感想}

液晶ディスプレイとプラズマディスプレイ、そして有機ELの違いを調べたが、その差は大きかった。
プラズマディスプレイは現在市場で衰退、液晶ディスプレイの一人勝ちといったものが散見された。対象的に有機ELは引用にもあったように主に小型端末分野でシェアを伸ばしている様子であった。特に2020年ではスマートフォンの分野において約50\%の割合を占める予測も出されていた。\cite{trend1}
プラズマテレビの世界初販売はパナソニックが握っていたこと\cite{trend2}を初めて知ったが、生産が終了し衰退してしまうのは少し残念に思った。
ものを売る際にその技術が後にどれだけ進歩するか、後方互換性があるのか、など先を見通せるかということが重要に感じた。

\begin{thebibliography}{99}
  \bibitem{sour} \verb+https://global.canon/ja/technology/s_labo/light/002/03.html+, 「キヤノン:技術のご紹介 | サイエンスラボ テレビと液晶」.
  \bibitem{trend1} \verb+https://news.mynavi.jp/article/20170708-amoled/, スマホの有機ELパネル採用率は2020年にはほぼ5割に到達 - TrendForceが予測 | マイナビニュース.+
  \bibitem{trend2} \verb+https://news.mynavi.jp/article/20131105-plasma/, なぜプラズマは主役になり得なかったのか? パナソニックのプラズマ撤退を検証する | マイナビニュース.+
\end{thebibliography}

\end{document}
