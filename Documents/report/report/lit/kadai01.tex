
\documentclass[uplatex]{jsarticle} %titlepage
\usepackage[dvipdfmx]{graphicx}
\usepackage{float}

\title{タッチパネルの方式とその利用法}
\author{C0118005 A3 秋本 遥基}
\date{}

\begin{document}
\maketitle

\section{もの及びその方式}
\subsection{iphone}

\begin{quotation}

iPhoneは静電容量方式と呼ばれるタッチパネル式。 
静電容量方式は、縦と横にセンサーが張り巡らされていて、
これらのセンサーは一定の電荷が保たれています。

\end{quotation}

以上は引用である。\cite{iphone}
%\begin{itemize}
%  \item
%\end{itemize}
静電容量方式は表面型と投影型の2種類があるが、iPhoneは投影型と思われる。\cite{source1}

\subsection{カーナビ}
panasonic社のgorillaEYEシリーズを例としてあげる。\cite{carnabi}
これは抵抗感圧式アナログタイプ (フィルム+ガラス)である。
また、この方式は抵抗膜方式とも呼ばれる。\cite{touchpanel}

%\begin{quotation}
%\end{quotation}

\begin{thebibliography}{99}
  \bibitem{iphone} \verb+https://iphonet.info/archives/1336, iphoneのタッチパネルの仕組みを分かりやすく書いてみた+
  \bibitem{source1} \verb+http://www.google.com/search?q=atom%3A%2F%2Fteletype%2Fportal%2Fea834e70-68df-469e-9287-fa96aa2a0e69, 次のiPhoneはタッチの強弱がわかる? タッチパネルの仕組みを知ろう|タブロイド|オトコをアゲるグッズニュース+
  \bibitem{carnabi} \verb+https://news.panasonic.com/jp/press/data/2015/05/jn150512-1/jn150512-1.html, SSDポータブルカーナビゲーション Gorilla EYEなど4機種を発売 | プレスリリース | Panasonic Newsroom Japan+
  \bibitem{touchpanel} \verb+https://www.eizo.co.jp/eizolibrary/other/itmedia02_08/, 第8回 なぜ画面に直接触って操作できるのか?――「タッチパネル」の基礎知識 | EIZO株式会社+
\end{thebibliography}
\end{document}
