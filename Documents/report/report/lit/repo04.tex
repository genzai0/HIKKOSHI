
\documentclass[uplatex]{jsarticle} %titlepage
\usepackage[dvipdfmx]{graphicx}
\usepackage{float}

\title{仮想現実と拡張現実の違い}
\author{C0118005 A3 秋本 遥基}
\date{}

\begin{document}
\maketitle


\section{仮想現実}

  仮想現実とは現代のVR(バーチャルリアリティ)のことを指す言葉で人口現実感とも呼ばれる。
  本質的にはコンピュータ上でもう一つの現実空間があるように仮想空間を作成し、その仮想空間内にいるかのように振る舞えるという概念である。
  暴力的だがARと大別できる点は仮想世界への没入という点でヘッドマウントディスプレイ(HMD)の装着、または現実空間で本来そこにあるべき空間の画像・動画の画面表示の有無である。(参考:\cite{source})

\section{拡張現実}

  拡張現実とは一般のAR(オーグメンテッドリアリティ)のことを指す言葉である。
  VRが仮想空間への没入に対して、ARは現実世界にデジタル情報を表現することを本質としている。内容としてはSF映画などの疑似ホログラムの表現に近い。
  カメラで撮った映像に3Dの物体またはキャラクターをあたかも、映像内の物理、空間法則に従うように表現したものが多い。(参考:\cite{source})

\section{感想}

  VR,AR技術(の表層)が好きで、学術的でないバラエティとしての記事を少し読んでいたがやはり面白いものが多いなと改めて感じた。VR,ARに加えて、ARをより拡張して媒体にとらわれずに、VRから回帰して現実世界との融合をという、MR(ミックスドリアリティ)技術があって、マイクロソフトのHoloLensを近年よく耳にする。要領としては、キーボードで操っていたものを、タブレット操作で、タブレットで操っていたものを擬似的な画面の操作で、といった具合に現実世界の拡張を主眼とした技術であると捉えているが、個人的には操作の煩雑さが減ってより視覚的なデバイス操作が可能なのと、3Dの情報・物体をより日常的で高度な表現ができると思っている。未だに技術的な問題と、スペック的、また価格的な問題で市井に流通はしていないが今後の発展を願うばかりである。

\begin{thebibliography}{99}
  \bibitem{source} \verb+https://vrinside.jp/news/vr-ar-mr/+
\end{thebibliography}

\end{document}
