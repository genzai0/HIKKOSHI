\documentclass[uplatex, titlepage]{jsarticle}
\usepackage[dvipdfmx]{graphicx}
\usepackage{float}

\title{宇宙と私の生き方の関係について}
\author{C0118005 A3 秋本 遥基}
\date{\today}

\begin{document}
\maketitle

\section{はじめに}

 本レポートでは宇宙と私の生き方の関係について、サイエンスの世界の授業(前期)を通して得た知見及び、科学的要素そして人生観に対する私見により考察するものである。

\section{本論}

 本講義(サイエンスの世界・前期)を受講し終わるまで、私は人類に対して漠然とした猜疑心を持っていた。第一の理由が、格差である。知恵を持つ現人類の祖が誕生するのが約200万年前である。しかし国や州、村などの堺による社会的不平等は依然として存在している。格差はどのように生まれているのか。私はこれについて人類の資本に対する確執、及び本能的なバイアスがもとであると考える。人類全体の理を考えた際、この格差は正すべきかどうかは自明である。しかし不審であるが故、または他に対しての競争性により人類は一つの集団になりえない。私の人類に対しての猜疑心は本来全ての人が平等になり、自己の利を追求することが愚かに思えるほどに完成された社会の創造がならないことである。\\
 サイエンスの世界において、私が最も印象に残ったものが一番目の講義である。宇宙観、思想にとらわれずにあくまで科学的に観察から漠然とした天体のありかたを定めていく。しかし、結果に対しての考察及び予測に恣意性があるかを疑い続けるその姿勢に感銘を受けた。他の役に立つこと、そして利を得て生活をすることが一般であった社会で、科学という現実には捉えにくく、更に生活という空間にその知識を落とし込むことが難しい学問を研究したのである(暦や一般的な時の巡りを定める、位置を知るなどの実益のあるものを逸脱してからも研究は続けられたことを示すものであり実益のない研究という意味ではない)。これは、確実な人類の進歩である。社会的な豊かさが、自由な知への探求を可能にしたのである。\\
 他にも、ブラックホールの観測などもとても興味深いものであった。光でさえ吸収してしまうため直接の観測ができず、しかしそのブラックホール周辺の星が落ちていく光の様相、
 以上より、サイエンスの世界の講義を受けて様々に有益な情報があった。我々が

\section{おわりに}


\end{document}
