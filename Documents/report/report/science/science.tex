\documentclass[uplatex, titlepage]{jsarticle}
\usepackage[dvipdfmx]{graphicx}
\usepackage{float}

\title{生物とこれからの生活}
\author{C0118005 A3 秋本 遥基}
\date{\today}

\begin{document}
\maketitle

\section{はじめに}

 本レポートは生物とこれからの人類の発展について、サイエンスの世界の授業(後期)を通して得た知見及び、科学的要素によって考察するものである。

\section{本論}

\subsection{先進的な細胞について}

  サイエンスの世界後半の講義について私はiPS細胞及びES細胞について興味をもったのでそれについて再び理解する。\\
  以下参考よりES細胞について述べる\cite{ES}。\\
  ES細胞とは、動物の発生初期に形成される胚の一部から作られる、多能性を持った細胞である。別名胚性幹細胞と呼ばれている。
  ES細胞の再生医療への応用として多能性を持ったES細胞から、人間の器官に属する細胞の機能をもつよう分化した細胞を再び人間に移植するという方法がある。
  しかし、この方法では分化する元の受精卵(ES細胞)と移植される側の人間とで拒絶反応が起こってしまうため、実用には至っていない。
  また、卵の核と体細胞の核を置換する方法も考えられているが成功率は低い。
  ES細胞は生体外で増殖させ続けると次第に染色体や遺伝子に異常が生じ、それが蓄積されていくことが明らかになっている。
  それ以外にも、ES細胞は受精卵を用いることによる倫理的問題や、卵と体細胞の核を入れ替えることによるクローン個体の作成による倫理的問題など、人権的な問題も多数提起されている。

以上、参考である。\\

  以下参考よりiPS細胞について述べる\cite{iPS}。\\
  iPS細胞とは体細胞に対して数種類の遺伝子を導入することにより、作成することのできる、多能性を持った細胞である。別名人工多能性幹細胞と呼ばれている。
  iPS細胞の再生医療への応用として、ES細胞と同様に人間の器官の働きをするよう分化させた細胞を再び人間に移植するものが有る。
  ES細胞との違いの一つは、倫理的問題の解決である。iPS細胞は体細胞から作成されるため、受精卵(受精済みの卵と精子を指し、単体としての精子、卵子ではない)を使うことがない。
  他にも拒絶反応にも違いが有る。別の細胞として作成されるES細胞とは違い、iPS細胞は患者自身の体細胞から作成されるため、拒絶反応が起こらない臓器、細胞が作成可能なのではないかと期待されている。
  また、再生医療の分野以外でも特定の病気に対しての薬剤の効果や毒性を調べたり、
  難病への対策として採取が難しかった病変組織の研究に採用されたりと、医学に対して多くの貢献をもたらすものと考えられる。
  しかし、一方で時間的コストがかかる方法でもある。\\

以上参考である。

\subsection{これからの医学とその発展について}

  iPS細胞とES細胞について比較した。これから、iPS細胞がどんどんと使われより多くの患者が健康な細胞、臓器で過ごせるようになればよいが、未だiPS細胞には問題が残っている。
  まず、挙げられるのがiPS細胞の癌化である。初期化因子の導入によって普通の体細胞から、iPS細胞の素となるものを作る時点でもともと染色体内にある遺伝子に変異が起こり、
  内在性発癌遺伝子を通常よりも活性化させてしまうのではないかと考えられてもいる。次に分化しきれないままに、万能性を有したまま移植されてしまい、
  移植された先でその細胞が奇形腫として表面化する問題である。これらの問題に対して様々な解決方法が考えられているが、それによってiPS細胞の作成に時間がかかったりと未だに
  問題点が残っている。\\
  しかし、iPS細胞のマウスへの移植実験については拒絶反応について起こらなかった事例も多く存在している。\\
  また、近年加齢黄斑変性や重症心不全などの一部の疾患について、臨床研究が行われている。さらにはマウスの実験で正常な精子・卵子を作成することが確認できた。
  これにより、不妊の原因究明にまで至るかもしれない。\\

\section{おわりに}

 iPS細胞とES細胞の違い、そしてこれから我々の生活がどのように変わっていくか考察した。漠然とした知識のままだが、これから移りゆく医療、生物学に少し関心が向いた。\\
  今回取り扱ったのはiPS細胞とES細胞であるが、日本からはSTAP細胞という不正認定を受けた研究も有る。真偽はともかく、研究者が正当に評価される世の中になってほしい。\\

\begin{thebibliography}{99}
  \bibitem{ES} \verb+https://ja.wikipedia.org/wiki/%E8%83%9A%E6%80%A7%E5%B9%B9%E7%B4%B0%E8%83%9E+
  \bibitem{iPS} \verb+https://ja.wikipedia.org/wiki/%E4%BA%BA%E5%B7%A5%E5%A4+\\
  \verb+%9A%E8%83%BD%E6%80%A7%E5%B9%B9%E7%B4%B0%E8%83%9E+

\end{thebibliography}

\end{document}
