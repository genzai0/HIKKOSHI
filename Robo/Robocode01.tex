\documentclass[a4j, titlepage]{jsarticle}
\usepackage[dvipdfmx]{graphicx}
\usepackage{float}
\usepackage{listings,jlisting}
\usepackage{url}
\lstset{%
  language={java},
  basicstyle={\small\ttfamily},%
  identifierstyle={\small},%
  commentstyle={\small\itshape},%
  keywordstyle={\small\bfseries},%
  ndkeywordstyle={\small},%
  stringstyle={\small\ttfamily},
  frame={tblr},
  breaklines=true,
  columns=[l]{fullflexible},%
  numbers=left,%
  xrightmargin=0zw,%
  xleftmargin=3zw,%
  numberstyle={\scriptsize},%
  stepnumber=1,
  numbersep=1zw,%
  lineskip=-0.5ex%
}

\pagestyle{plain}
\title{コンピュータサイエンス基礎実験 テーマ1 レポート}
\author{学籍番号 C0118005\\氏名 秋本 遥基\\メールアドレス c011800588@edu.teu.ac.jp}

\date{実験日 2019/06/11\\提出日 2019/06/18}


\begin{document}

\begin{titlepage}
\maketitle
%\thispagestyle{empty}
\end{titlepage}

%\addcontentsline{page}{-1}

\section{目的}

このレポートではRobocodeにおいて基本的な動きをまとめたMyFirstRobotクラスについて考察する。

\section{ソースコード}
以下の\ref{list}にソースコードを示す。

\lstinputlisting[caption=MyFirstRobot.java, label=list]{src/MyFirstRobot.java}

\section{解説}

二章のMyFirstRobotクラスについて解説する。

\subsection{void run()メソッド}
ここではRobotの画面上での動きを定めている。turnGunRight(),turnGunLeft()でロボットの大砲を引数の角度分右または左に回転させ、ahead(),back()で関数内分の移動を前または後ろに引数分移動する。
また、turnGunRight(),turnGunLeft()関数は大砲とレーダーを同期して回転し、レーダーのみを回転させる場合はturnRaderRight(),turnRaderLeft()関数を仕様する。
したがってここでは、100前進、右回りに敵を探す、100後退、左回りに敵を探す、100前進…というように無限ループをする。

\subsection{void onScannedRobot()メソッド}
ここではRobotのレーダーの範囲内に敵が存在した場合の処理を行う。
fire()関数は引数分のパワーを持った弾丸を放つ関数。引数の有効範囲は.1から3までで、引数が1以下の場合、「引数*4」のダメージを、それより多い数を引数にとった場合「引数*4+2*(引数-1)」のダメージを与える。
また、この弾丸について「敵に当たったとき」にonBulletHit()が、「弾丸に当たったとき」にonBulletBullet()が、「壁に当たったとき」にonBulletMissed()がそれぞれイベントとして作成される。

\subsection{void onHitByBullet()メソッド}
ここでは自機に弾丸が当たった場合の処理を行う。
back()は2.1で先述した通りである。

\subsection{void onHitWall()メソッド}
ここでは自機が移動した際に壁に激突した場合の処理を行う。
back()は2.1で先述した通りである。

\section{考察}

基本動作について解説をし、結果として示された基礎的なturnGunRight()関数などとturnRaderRight()関数などの
違いは今後のRobocodeをするにあたり、偏差射撃の実装などに大いに役立つと思われる。

今回作ったのは非常に簡単な動きをするものであり、勝率も芳しくはないが
レーダーによる索敵及び攻撃、被弾による場所移動(回避行動)など最低限の動作はしている。

\begin{thebibliography}{9}
    \bibitem{RobocodeAPI}
    Robot (Robocode API ドキュメンテーション),
    \url{http://www.solar-system.tuis.ac.jp/Java/robocode_api/robocode/Robot.html},
    参考日時(2019/06/17)
\end{thebibliography}

\end{document}